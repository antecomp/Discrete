\documentclass[12pt, letterpaper]{article}

% usepackage aids
	
	% for graphics
	\usepackage{graphicx}
	\graphicspath{{images/}} %configuring the graphicx package

	% to change the margin sizes
	\usepackage[margin=2.7cm]{geometry}

	% assorted maths packages
	\usepackage{xfrac,bigints}

	% handles maths better than latex default
	\usepackage{amsmath}

	% for wrapping text around images
	\usepackage{wrapfig}% http://ctan.org/pkg/wrapfig

	% colour tables
	\usepackage[table]{xcolor}

	% idk
	\usepackage{wasysym}

	% custom margins
	\usepackage{scrextend}

	% for symbols and glyphs
	\usepackage{fontawesome}

	% adds the fancy fun margin notes
	\setlength {\marginparwidth }{2cm}
	\usepackage[colorinlistoftodos]{todonotes}

	% Add linebreaks without indentation because you need a package for that too????
	\usepackage[parfill]{parskip}

	% for highlighting text
	\usepackage{soul}
		% make multicolour not aids
		\newcommand{\hlc}[2][yellow]{{\sethlcolor{#1}\hl{#2}}}

	% ability to mess with bullet list indents
	\usepackage{enumitem}


	% subtable maybe
	\usepackage{booktabs,subcaption,amsfonts,dcolumn}


	% Better underlines
	\usepackage{contour}
	\usepackage[normalem]{ulem}

	\renewcommand{\ULdepth}{1.8pt}
	\contourlength{0.8pt}

	\newcommand{\cul}[1]{%
		\uline{\phantom{#1}}%
		\llap{\contour{white}{#1}}%
	}

	% monospace font
	\usepackage{courier}
	\usepackage[T1]{fontenc}

	% images
	\usepackage{graphicx}
	\graphicspath{ {./images/} }

	% line breaks in tables
	\usepackage[usestackEOL]{stackengine}

	% math symbols
	\usepackage{amssymb}


	% ???
	\usepackage{tabularx}



	%%%%%%%%%%%%%%%%%%%%%%%%%%%%%%%%%%%%%%%%%%%%%%%%%%%%%%%%%%%%%%%%%%%%%%%%%%%



	% personal newcommands to make a lot of my formatting easier.
	\newcommand{\exheader}[1][ex]{{\tiny{#1}\normalsize}}

	\newcommand{\sidenote}[1][INSERT TEXT HERE]{{\todo[size=tiny, color=lightgray, nolist]{\faInfoCircle\space {#1}}}}

	\newcommand{\intextnote}[1][Also Note Lorem Ipsum]{{
				\emph{
					\begin{small}
					\begin{quote}
						\faInfoCircle\space
						{#1}
					\end{quote}
					\end{small}
		}
	}}

	\newcommand{\horizline}[0]{\noindent\rule{\textwidth}{1pt}\\}


	% for equiv tables
	\newcommand{\eqtbitem}[3][----]{ {#1} & $ \equiv $ & {#2} & {#3} }

	% def
	\newcommand{\define}[1][-----]{\horizline
	\faBook \space {#1} \\
	\horizline}


	% change inner list label
	\renewcommand{\labelitemii}{$\rightarrow$}


\title{Discrete Maths Notes}
\begin{document}
\maketitle
\tableofcontents
\pagebreak


\section{Logic}
\bigbreak\bigbreak
\subsection{Propositional Logic}
\bigbreak
\subsubsection{Basics}

\indent
\todo[size=tiny, color=lightgray, nolist]{
	\faInfoCircle\space It's not always easy to determine if they're true/false.
}A \textbf{proposition} is a statement that is either true or false. \\
Prepositions will be represented mathematically with capital letters A, B, C... \\ These prepositions are then are connected into more complex compound prepositions using \emph{connectives}. Connectives are statements like "and, implies, if-then" and are represented mathematically with the symbols below.

\begin{center}
\begin{tabular}{ |p{1.5cm}|p{3.75cm}|p{5.8cm}|p{3.25cm}| }
 \hline
 \multicolumn{4}{|c|}{Connectives} \\
 \hline
 \hline
 \rowcolor{lightgray} Symbol& Name & English Term(s) & Reading \\
 \hline
 	$\land$ 	& 	AND 		& And, But, Also	& A and B  	\\
	\hline
 	$\lor$ 		& 	OR 			& -	& A or B	\\
	\hline
 	$\implies$ 	& 	IMPLICATION	& \begin{itemize}[leftmargin=*, label={}]
		\item If A, Then B
		\item If A, then B
	 	\item A implies B
		\item A, therefore B
		\item A only if B
		\item B follows from A
		\item A is a sufficient condition for B
		\item B is a necessary condition for A 
	\end{itemize} & A implies B \\
	\hline
	$\iff$ 		&	BICONDITIONAL	& If \& only if	 A is necessary and sufficient for B & A if and only if B 	\\
	\hline
	$\neg$ 		&	NEGATION		& Not... & Not A \\
 \hline	
\end{tabular}
\end{center}

\emph{
\begin{small}
\begin{quote}
	\faInfoCircle\space
	A Biconditional can also be thought of $(A \implies B)\land(B \implies A)$ \\
	Negation may sometimes be represented as $A\prime \textrm{ or } \overline{A}$
\end{quote}
\end{small}
}

\bigbreak

\subsubsection{Terminology}
\begin{addmargin}[2em]{2em}
	$ A \land B $ - \cul{conjunction} of \cul{conjuncts} $A$ and $B$ \\
	$ A \lor B $ - \cul{disjunction} of \cul{disjuncts} $A$ and $B$ \\
	$ A \implies B $ - $A$ is the \cul{hypothesis\tiny{/antecedent}} and $B$ is the \cul{conclusion\tiny{/consequence}}  
\end{addmargin}


\pagebreak
\subsubsection{Examples}

\exheader[1 Compound Proposition]
\begin{addmargin}[1.5em]{1.5em}
	If \cul{all humans are mortal}\tiny{prp $A $}\normalsize \space and \cul{all Greeks are human}\tiny{prp $B$}\normalsize\space
	\\then \cul{all Greeks are mortal}\tiny{prp $C$}\normalsize\space can be represented as $A \land B \implies C$
\end{addmargin}

\exheader[2 Negation]
\begin{addmargin}[1.5em]{1.5em}
	\begin{itemize}[label={}, leftmargin=*]
		\item Chocolate is sweet $\rightarrow$ Chocolate is \cul{not} sweet
		\item 
			Peter is tall and thin $\rightarrow$ 
			\todo[size=tiny, color=lightgray, nolist]{
				\faInfoCircle\space Short \underline{and} fat would be incorrect!
			}
			Peter is short \cul{or} fat
		\item The river is shallow or polluted $\rightarrow$ 
		\todo[size=tiny, color=lightgray, nolist]{
			\faInfoCircle\space Not shallow \underline{or} not polluted would be incorrect!
		}
		The river is deep \cul{and} polluted.
	\end{itemize}
\end{addmargin}

\exheader[{3 Implication: \hlc[cyan]{hypothesis} and \hlc[green]{conclusion} }]
\begin{addmargin}[1.5em]{1.5em}
	\begin{itemize}[label={}, leftmargin=*]
		\item If \hlc[cyan]{the rain continues} then \hlc[green]{the river will flood}
		\item A sufficient condition for a \hlc[green]{network failure} is that the \hlc[cyan]{central switch goes down}
		\item \hlc[cyan]{The avocados are ripe} only if \hlc[green]{they are dark and soft}
		\item \hlc[green]{A good diet} is a necessary condition for \hlc[cyan]{a healthy cat}
	\end{itemize}
\end{addmargin}

\bigbreak
\subsubsection{Satifiability, Tautology, Contradiction}
\begin{itemize}[label={}, leftmargin=*]
	\item A proposition is \cul{satisfiable} if it is true for
	\sidenote[You don't need a whole truth table for this, just look for one!]\emph{at least one} 
	combination of boolean values.
	\item \cul{A Boolean Satisfiability Problem (SAT)} is checking for satisfiability in a propositional logic formula.
	\item \cul{A Tautology} is a proposition that is \cul{always true} \\ \hspace*{5mm} \exheader[ex] $ A \lor \neg A$
	\item \cul{A Contradiction} is a proposition that is always false. \\ \hspace*{5mm} \exheader[ex] $ A \land \neg A$
\end{itemize}

\pagebreak

\subsection{Truth Tables}
\bigbreak
\subsubsection{Basics}
Truth Tables are used for determining all the possible outputs of a complex compound propostion.
\\
\horizline
\\
\cul{The Columns} Are for the prepositions,
\sidenote[The intrmt' prepositions are optional steps to make solving easier, use as needed.]intermediate compound prepositions and the whole compound preposition.

\cul{The Rows} Are to contain the different sets of possible truth values for each proposition. You will have $2^p$ rows where $p$ is the number of propositions (then +1 for the header).

\horizline \\
\faWarning \space The connectives in a compound propositional logic problem follow an order of precedence (the PEMDAS of logic) in the following order; \\
$ \neg $ , $ \land $ , $ \lor $ , $ \implies $ , $ \iff $
\\
\subsubsection{Connective Outputs}
\begin{table}[h]
    \begin{subtable}[h]{0.20\textwidth}
        \centering
        \begin{tabular}{l | l}
		\multicolumn{2}{c}{\textbf{Negation}} \\
        \hline \hline
        $A$ & $\neg A$ \\
		\hline
        T & F \\
		F & T
       \end{tabular}
    \end{subtable}
    \hfill
    \begin{subtable}[h]{0.20\textwidth}
        \centering
        \begin{tabular}{l | l | l}
		\multicolumn{3}{c}{\textbf{And}} \\
        \hline \hline
		$A$ & $B$ & $A \land B$ \\
		\hline
		T & T & T \\
		T & F & F \\
		F & T & F \\
		F & F & F 
        \end{tabular}
     \end{subtable}
	 \hfill
	 \begin{subtable}[h]{0.20\textwidth}
        \centering
        \begin{tabular}{l | l | l}
		\multicolumn{3}{c}{\textbf{Or}} \\
        \hline \hline
		$A$ & $B$ & $A \lor B$ \\
		\hline
		T & T & T \\
		T & F & T \\
		F & T & T \\
		F & F & F 
        \end{tabular}
     \end{subtable}

\end{table}



\begin{table}[h]
    \begin{subtable}[h]{0.45\textwidth}
        \centering
        \begin{tabular}{l | l | l}
		\multicolumn{3}{c}{\textbf{Implication}} \\
        \hline \hline
		$A$ & $B$ & $A \implies B$ \\
		\hline
		T & T & T \\
		T & F & F \\
		F & T & T \\
		F & F & T 
        \end{tabular}
		\caption*{\small{An implication is true when the hypothesis is false or the conclusion is true.}}
     \end{subtable}
    \hfill
    \begin{subtable}[h]{0.45\textwidth}
        \centering
        \begin{tabular}{l | l | l}
		\multicolumn{3}{c}{\textbf{Biconditional}} \\
        \hline \hline
		$A$ & $B$ & $A \iff B$ \\
		\hline
		T & T & T \\
		T & F & F \\
		F & T & F \\
		F & F & T 
        \end{tabular}
		\caption*{\small{A Biconditional is true when the two prepositions have the same value.}}
     \end{subtable}
\end{table}

Out of all these outputs, the most unintuitive is the 3rd implication output ($F, T \implies T$). The easiest way to understand this output is with the proposition ``If it is raining, then the ground is wet''; now say you step outside and it is not raining, but the ground is wet. The entire statement isn't false or incorrect, but the first part of it still has a false value. \\ The only way to make an implication false is when the hypothesis is true but the conclusion is false.


\pagebreak
\subsubsection{Examples}

\begin{table}[h]
    \begin{subtable}[h]{0.45\textwidth}
        \centering
        \begin{tabular}{l | l | l | l | l}
		\multicolumn{5}{c}{\textbf{$A \implies B \iff B \implies A$}} \\
        \hline \hline
		$A$ & $B$ & $A \implies B$ & $ B \implies A $ & ------ \\
		\hline
		T & T & T & T & T \\
		T & F & F & T & F \\
		F & T & T & F & F \\
		F & F & T & T & T
        \end{tabular}
     \end{subtable}
    \hfill
    \begin{subtable}[h]{0.45\textwidth}
        \centering
        \begin{tabular}{l | l | l | l | l}
		\multicolumn{5}{c}{\textbf{$A\land \neg B \implies \neg C$}} \\
        \hline \hline
		$A$ & $B$ & C & $A \land \neg B $ & --------  \\
		\hline
			T & T & T & F & T \\
			T & T & F & F & T \\
			T & F & T & T & F \\
			T & F & F & T & T \\
			F & T & T & F & T \\
			F & T & F & F & T \\
			F & F & T & F & T \\
			F & F & F & F & T
        \end{tabular}
     \end{subtable}
\end{table}
\intextnote[Remember, columns like $A \implies B$ are optional in-between steps to help solve each problem.]
\bigbreak
\subsubsection{Exercise: Finding Tautologies, Satisfiable \& Contradicting Props'}

Indicate whether each of the following is a tautology, satisfiable but not a tautology or a contradiction;
\begin{enumerate}[leftmargin=*, label={}]
	\item $ A \implies B $
	\item $ A \implies A $
	\item $ A \implies \neg B \lor \neg C $
	\item $ A \lor B \implies B $
	\item $ (A \land B) \implies (A \lor B) $
	\item $ A \lor \neg A \implies B \land \neg B $
\end{enumerate}

\small{(Answers and explanations on the next page...)} \normalsize

\pagebreak

\intextnote[Notice how none of these rely on drawing out a whole truth table! Focus on trying to find a way to get each proposition to output true and a way to get it to output false!]

$ A \implies B $ \\
\emph{Satisfiable but not a tautology} \\
Just knowing the properties of an implication you should know there's way to get true outputs and a false output.

$ A \implies A $ \\
\emph{Tautology} \\
Only would be $ T \implies T $ or $ F \implies F $, both of which result in true.

$ A \implies \neg B \lor \neg C $ \\
\emph{Satisfiable but not a tautology} \\
Instead of making a long unpleasant truth table, it's easiest to start by simply looking for one true and one false possible output. \\
We can make the left side true simply by making A false, since all that remains is an or statement we now have a true output. \\
We can just as easily find a false output for this proposition with $A=T$, $B=T \space (\neg B = F)$ to make the implication false, then we can just make $ \neg C $ false to make the or output false.

$A \lor B \implies B $ \\
\emph{Satisfiable but not a tautology} \\
If we make B true then the biconditional will always be true regardless of A. \\
There is only one way to make an implication false, so if we can set up A and B to result in that false output, it won't be a tautology. If we make A true and B false it will make the implication false!

$ (A \land B) \implies (A \lor B) $ \\
\emph{Tautology} \\
Remember the only way to make an implication false is if the hypothesis is true and the conclusion is false. There is absolutely no way to do this because of the and/or setup! 

$ A \lor \neg A \implies B \land \neg B $ \\
\emph{Contradiction} \\
The left side is always true and the right side is always false. So the result of the implication is always false!
\pagebreak

\subsection{Equivalence}
\bigbreak
\subsubsection{Introduction to Equivalence}

\horizline
\faBook \space Two (compound) propositions $P$ and $Q$ are \textbf{logically equivalent} when their truth values always match (Meaning they'll have the same truth table!). \\
Equivalence is denoted by $P \equiv Q$.\\
\horizline
\\
Equivalence relates heavily to the concept of Tautologies;
\begin{itemize}[label={}, leftmargin=*]
	\item $P$ and $Q$ are equivalent when $P \iff Q$ is a tautology.
	\item A proposition $P$ is a tautology iff (if and only if) it is equivalent to $T$ (true), i.e $P \equiv T$
\end{itemize}

\bigbreak

\subsubsection*{Examples}
\exheader[1]\\
Given the implication $A \implies B$, are the following equivalent?
\begin{itemize}[leftmargin=*, label={}]
	\item The contrapositive: $\neg B \implies \neg A$
	\item The converse: $B \implies A$
\end{itemize}
\bigbreak
\begin{center}
\begin{tabular}{l | l || l | l | l}
	\hline 
	$A$ & $B$ & $A \implies B$ & $ \neg B \implies \neg A $ & $B \implies A$ \\
	\hline
	T & T & T & T & T \\
	T & F & F & F & T \\
	F & T & T & T & F \\
	F & F & T & T & T
	\end{tabular} 
\end{center} \bigbreak
Looking at the table we can see that $A \implies B$ and $\neg B \implies \neg A$ are equivalent.
\bigbreak


Now, what about $\neg A \lor B $ ? \\
\begin{center}
\begin{tabular}{l | l || l | l}
	\hline
	$A$ & $B$ & $A \implies B$ & $\neg A \lor B $ \\
	\hline
	T & T & T & T \\
	T & F & F & F \\
	F & T & T & T \\
	F & F & T & T \\
\end{tabular}
\end{center} \bigbreak
\indent \sidenote[This is actually one of the equivalence laws you'll see in the next \\ section!]Yep! $\neg A \lor B \equiv A \implies B$.

\pagebreak


\exheader[2: Code Logic Optimization]

Understanding equivalent boolean expressions is very important in computer science (for code) and chip design (for logic gates). Consider the code below;
\begin{verbatim}
	if(x > 0 || (x <= 0 && y > 100))
\end{verbatim}
Lets see if we can change this expression to something equivalent but simplified. \\
Let $A$ be \texttt{x > 0} and let $B$ be \texttt{y > 100} \\
Now we can compare the truth values of $ A \lor (\neg A \land B) $ and $ A \lor B $. \\
\begin{center}
	\begin{tabular}{l | l || l | l}
		\hline
		$A$ & $B$ & $A \lor (\neg A \land B)$ & $ A \lor B $ \\
		\hline
		T & T & T & T \\
		T & F & T & T \\
		F & T & T & T \\
		F & F & F & F \\
	\end{tabular}
\end{center} \bigbreak
They're equivalent! We can reduce the if statement's expression to simply;
\begin{verbatim}
	if(x > 0 || y > 100)
\end{verbatim}

\pagebreak
\subsubsection{Equivalence Laws}
For more complex propositions it is impractical to create a set of massive truth tables to check for equivalence. So instead we utilize equivalence laws to directly prove equivalence.
\bigbreak
\subsubsection*{Nine Equivalence Laws;}
\small{\emph{Many of these are pretty self-explanatory}}\normalsize
\begin{itemize}[label={}, leftmargin=*]
	\item Double Negation Law: $\neg(\neg A) \equiv A$ 
	\item Identity Laws: $ A \land T \equiv A $ \hspace*{1.5em} $ A \lor F \equiv A$
	\item Domination Laws: $ A \lor T \equiv T $ \hspace*{1.5em} $ A \land F \equiv F $
	\item Commutative Laws: $ A \land B \equiv B \land A$  \hspace*{1.5em} $ A \lor B \equiv B \lor A $
	\item Associative Laws: $ (A \land B) \land C \equiv A \land (B \land C) $ \hspace*{1.5em} $(A \lor B) \lor C \equiv A \lor (B \lor C) $
	\item Idempotent Laws: $ A \land A \equiv A $ \hspace*{1.5em} $ A \lor A \equiv A$
	\item\sidenote[Very similar to the algebriac distributive law]Distributive Laws: $A \lor (B \land C) \equiv (A \lor B) \land (A \lor C) $ \hspace*{1.5em} $A \land (B \lor C) \equiv (A \land B) \lor (A \land C) $
	\item DeMorgan's Laws: $\neg (A \land B) \equiv \neg A \lor \neg B $ \hspace*{1.5em} $\neg (A \lor B) \equiv \neg A \land \neg B$
	\item Implication Laws: $ A \implies B \equiv \neg B \implies \neg A \equiv \neg A \lor B$
\end{itemize}
\todo[inline]{TODO: Maybe refomat this as a table so its a bit easier to quickly reference?}
% in solving these, our goal is to reduce the # of letters in the props. - do the side with more stuff going on :)

\pagebreak

\subsubsection*{Examples}
\exheader[1]
Prove $A \lor (\neg A \land B) \equiv A \lor B$ \\ \break
\begin{tabular}{r  l  l  l}
	\eqtbitem[$ A \lor (\neg A \land B) $]{$(A \lor \neg A)\land(A \lor B)$}{(Distributive)} \\
	\eqtbitem[]{$T \land (A \lor B)$}{} \\
	\eqtbitem[]{$A \lor B$}{(Identity)}
\end{tabular}
\intextnote[In solving these, the goal should be to reduce the \# of letters in the propositions. Focus on the side of an equivalence with more going on and try to reduce it down since the more complex proposition will have more oppurtunities to utilize the different equivalence laws. ]
\bigbreak\bigbreak


\exheader[2]
Simplify $ A \land \neg(A \land B) $ \\
\begin{tabular}{r  l  l  l}
	\eqtbitem[$ A \land \neg(A \land B) $]{$A \land (\neg A \lor \neg B)$}{(DeMorgan's)} \\
	\eqtbitem[]{$(A \land \neg A) \lor (A \land \neg B)$}{(Distributive)} \\
	\eqtbitem[]{$F \lor (A \land \neg B)$}{} \\
	\eqtbitem[]{$A \land \neg B$}{(Identity)}
\end{tabular}
\intextnote[You don't have to name the laws you're using in the homework, the simple $\equiv$ down the middle format for each step is fine.]
\bigbreak\bigbreak

\exheader[3] Show that $(A \land B) \implies (A \lor B)$ is a tautology. \\
\begin{tabular}{r  l  l  l}
	\eqtbitem[$(A \land B) \implies (A \lor B)$]{$\neg (A \land B) \lor (A \lor B)$}{(Implication)} \\
	\eqtbitem[]{$(\neg A \lor \neg B)\lor (A\lor B)$}{(DeMorgan's)} \\
	\eqtbitem[]{$\neg A \lor \neg B \lor A \lor B$}{(Associative)} \\
	\eqtbitem[]{$\neg A \lor A \lor \neg B \lor B$}{(Commutative)} \\
	\eqtbitem[]{$(\neg A \lor A) \lor (\neg B \lor B)$}{(Associative)} \\
	\eqtbitem[]{$T \lor T$}{} \\
	\eqtbitem[]{$T$}{(Idempotent)}
\end{tabular}

% Todo: work on excercises and add them! (try them yourself before simply copying down!!)

\pagebreak

\subsection{Arguments}
\bigbreak
\define[An \textbf{argument} is a sequence of propositions in which the conjunction of the initial propositions implies the final proposition \\
An argument can be represented as; \\ $P_1 \land P_2 \land P_3 ... \land P_n \implies Q$]
\subsubsection*{Examples}
\begin{itemize}[leftmargin=*, label={}]
	\item If George Washington was the first president of the United States, then John Adams was the first vice president. George Washington was the first president of the United States.  Therefore John Adams was the first vice president.
		\begin{itemize}
			\item Let A be  “George Washington was the first president of the United States.”
			\item Let B be "John Adams was the first vice president.”
			\item $(A \implies B) \land A \implies B$
		\end{itemize}
	\item If Martina is the author of the book, then the book is fiction.  But the book is nonfiction. Therefore Martina is not the author.
		\begin{itemize}
			\item Let A be “Martina is the author of the book.”
			\item  Let B be “The book is fiction.”
			\item $(A \implies B) \land \neg B \implies \neg A$
		\end{itemize}
	\item The dog has a shiny coat and loves to bark. Consequently, the dog loves to bark.
		\begin{itemize}
			\item Let A be “The dog has a shiny coat.”
			\item Let B be “The dog loves to bark."
			\item $A \land B \implies B $
		\end{itemize}
\end{itemize}
\pagebreak
\subsubsection{Valid Arguments / Inference Rules}
\bigbreak
\define[An argument is \textbf{valid} if and only if \textbf{its conclusion is never false while its premises are true.}] \smallbreak
We can't use a truth table to validate an argument since it only shows the truth values for the statement as a whole, instead we need to use new \textbf{Inference Rules}
\smallbreak

%todo: type these out nicely instead of using images. - try using \atop 
\subsubsection*{Inference Rules}
% $\frac{P \atopP \implies Q}{\therefore Q}$
%\begin{center}
%	\includegraphics[width=8cm]{if1}
%	\includegraphics[width=16cm]{if2}
% \end{center}


% I wasted so much time making this lol
\bigbreak
\begin{center}
	\begin{tabular}{c c}
		\Longunderstack{
			$
			{
				\begin{array}{l}
					P \\
					P \implies Q \\
					\hline
					\therefore Q
				\end{array}
			}
			$ \\ \\
			{\tiny Ex: If George Washington...}
		} 
		& \Longunderstack{
			$
			{
				\begin{array}{l}
					P \implies Q \\
					\neg Q \\
					\hline
					\therefore \neg P
				\end{array}
			}
			$ \\ \\
			{\tiny Ex: If Martina...}
		} \\ \\ \\
		\Longunderstack{
			$
			{
				\begin{array}{l}
					P \land Q \\
					\hline
					\therefore P
				\end{array}
			}
			$
		} 
		& \Longunderstack{
			$
			{
				\begin{array}{l}
					P \\
					\hline
					\therefore P \lor Q
				\end{array}
			}
			$
		} \\ \\
		\multicolumn{2}{c}{
		\Longunderstack{
			$
			{
				\begin{array}{l}
					P \\
					Q \\
					\hline
					\therefore P \land Q
				\end{array}
			}
			$ \\ \\
			{\tiny Ex: Paul is a good swimmer. Paul is a good runner.} \\ {\tiny Therefore Paul is a good swimmer and a good runner}
		}}
	\end{tabular}
\end{center}

\intextnote[Each line of these rules are basically "if this prop is true and if that prop is true then the last prop is true"]

\pagebreak

\subsubsection*{Examples (finding conclusions)} 
\begin{itemize}
	\item If the car was involved in the hit-and-run, then the paint would be chipped. But the paint is not chipped.
		\begin{itemize}
			\item "Car was involved in a hit-and-run" $\rightarrow P$
			\item "Paint would be chipped" $\rightarrow Q$
			\item "The paint is not chipped" $\rightarrow \neg Q$
			\item Conclusion: The car was not involved in a hit-and-run. From the second rule!
		\end{itemize}
	\item If the bill was sent today, then you will be paid tomorrow. You will be paid tomorrow.
		\begin{itemize}
			\item Nothing can be concluded from this. \smiley
		\end{itemize}
	\item If the program is efficient$_P$, it executes quickly$_Q$. Either the program is efficient$_P$, or it has a bug$_R$. However, the program does not execute quickly$_{\neg Q}$.
		\begin{itemize}
			\item "If the program is efficient" $\rightarrow P$
			\item "it executes quickly" $\rightarrow Q$
			\item "it has a bug" $\rightarrow R$
			\item "the program does not execute quickly" $\rightarrow \neg Q$
			\item We start by knowing $P \implies Q$ and $P \lor R$ and $\neg Q$...
			\item $(P \implies Q)$ and $\neg Q$ can imply $\neg P$
			\item We need to transform $P \lor R$ to use it: $P \lor R \equiv \neg (\neg P) \lor R \equiv \neg P \implies R$
			\item $\neg P \implies R$ and $\neg P$ (the first implication we isolated) now implies $R$ by the first inference rule.
		\end{itemize}
\end{itemize}

\pagebreak

\subsubsection{Proving a Valid Argument}
Assuming the premises are true, apply a sequence of premises and derivation rules, which include the equivalence laws and inference.
\bigbreak
\textbf{General Steps} \\

\begin{enumerate}
	\item Identify all the premises (might need some transformations).
	\item \sidenote[Start with the RHS of the argument on the bottom of the list and work your way up] Think backwards. Start from what you want and then seek supporting premises, current results, and necessary equivalence laws and inference rules, until you reach the given premises.
	\item Write the proof sequence, where {\color{red}each step is either one premise or derived from previous step(s) using equivalence laws or inference rules.}
	
\end{enumerate}

\bigbreak

\subsubsection*{Examples}

\exheader[1]
Prove $(A \implies B) \land (\neg C \lor A) \land C \implies B$
\intextnote[This one is already in its standard form - so we just need to identify each part of the standard $P_1 \land P_2 \land P_3 ... \land P_n \implies Q$ form. At the end of this we want to prove that B is true.]

\begin{tabular}{r  l  l  l  l  l}
	$(A \implies B)$ & $\land$ & \Longunderstack{$(\neg C \lor A)$ \\ \tiny Implication Law} & $\land$ & $C$ & $\implies B$ \\
	$\downarrow$ & $\land$ & \Longunderstack{$C \implies A$ \\ \tiny first inference rule law} & $\land$ & $C$ & $\downarrow$ \\
	$(A \implies B)$ & $\land$ & \multicolumn{3}{l}{\Longunderstack{$A$ \\ \tiny also by first inf rule}} & $\downarrow$ \\
	\multicolumn{5}{c}{$B$} & $\implies B$
\end{tabular}

\intextnote[This is not usually how you would format these proofs, this table was to give you an idea of the actual process. The actual proof would look like the following;
\begin{enumerate}
	\item $A \implies B$
	\item $\neg C \lor A$
	\item $C$ 
	\item $C \implies A$ \hspace*{2cm} (2, Implication)
	\item $A$ \hspace*{3.3cm} (3,4)
	\item $B$ \hspace*{3.2cm} (1,5)
\end{enumerate}
You need to put every step in a seperate (numbered) line, starting with each component of the argument and then the transformations you do with the reason given. You dont need to name the law used but you need to mention the steps you combined to acheive the next part.]

\pagebreak

\exheader[2]
Prove $A \land (B \implies C) \land ((A \land B) \implies (D \lor \neg C)) \land B \implies D$
\intextnote[For this one focus on step 3 ($D \lor \neg C$) as your point to figure out this argument since its the only portion that has D in it.]
\begin{enumerate}
	\item $A$
	\item $B \implies C$
	\item $(A \land B) \implies (D \lor \neg C)$
	\item $B$
	\item $A \land B$ \hspace*{2.2cm} (1,4)
	\item $D \lor \land C$ \hspace*{1.9cm} (3,5)
	\item $C \implies D$ \hspace*{1.5cm} (6, Commutative, Implication) - {\tiny Communicative used to swap C, D}
	\item $C$ \hspace*{3cm} (2,4)
	\item $D$ \hspace*{3cm} (7,8)
\end{enumerate}
\todo[inline]{TODO: ADD MORE NOTES TO THIS ONE, I GENIUNELY AM NOT FOLLOWING THE PROCESS HERE}

\bigbreak

\exheader[3] Prove $(A \implies B) \land (\neg C \lor A) \land C \implies A \land B$
\intextnote[For this one notice that the right-hand side isn't a single letter anymore. We now need to focus on proving the whole $A\land B$ statement. So this problem is actually solved a bit backwards, start by writing the last steps ($A, B, A\land B$) and then go up and figure out how you can prove A.]
\begin{tabular}{c l l}
	1. & $A \implies B$ & \\
	2. & $\neg C \lor A$ & \\
	3. & $C$ & \\
	4. & $C \implies A$ & (2, Implication) \\
	5. & $A$ & (3,4) \\
	6. & $B$ & (1,5) \\
	7. & $A \land B$ & (5,6)
\end{tabular}

\intextnote[If instead this problem was looking for $A \lor B$, you could just prove either A or B to make the entire statement valid.]

\pagebreak


\subsubsection{The Deduction Method}

Now, what if the conclusion is in implication form? \\
There are two ways of solving for this form, the main one being \textbf{The Deduction Method...} \\

Suppose the argument has the form:
\begin{center}
	$ P_1 \land P_2 \land P_3 ... \land P_n \implies (R \implies S) $
\end{center}
where the conclusion itself is an implication. \cul{We can add R as an additional premise and then imply S}. In other words, we can have the argument:
\begin{center}
	$ P_1 \land P_2 \land P_3 ... \land P_n \land R \implies S $
\end{center}

\bigbreak \bigbreak

\subsubsection*{Examples}

\exheader[1] Prove $(A \implies B) \land (B  \implies C) \implies (A \implies C)$
\intextnote[Start with C at the bottom. Now the only way to validate C is if B is true, so make B step 4 and find the proper relations to make B true.]
Deduction: $(A \implies B) \land (B \implies C) \land A \implies C$ \\
\begin{tabular}{c l l}
	1. & $A \implies B$ & \\
	2. & $B \implies C$ & \\
	3. & $A$ & \\
	4. & $B$ & (1,3) \\
	5. & $C$ & (2,4) \\
\end{tabular}

\bigbreak

\exheader[2] Prove $\neg (A \land \neg B) \land (B \implies C) \implies (A \implies C)$ \\
Deduction: $\neg (A \land \neg B) \land (B \implies C) \land A \implies C$ \\
\begin{tabular}{c l l}
	1. & $\neg (A \land \neg B) $ & \\
	2. & $ B \implies C $ & \\
	3. & $A$ & \\
	4. & $\neg A \lor B $ & (1, DeMorgan's) \\
	5. & $A \implies B$ & (4, Implication) \\
	6. & $B$ & (3,5) \\
	7. & $C$ & (2,6)
\end{tabular}

%todo: excersise slide

\pagebreak

\subsection{Predicate Logic}



\define[A \textbf{predicate} represents the properties of/relations among objects. \\
Examples:
\begin{itemize}
	\item n \emph{is a perfect square}
	\item x \emph{is greater than} y
\end{itemize} \indent]

\subsubsection*{Often propositional logic is not enough!} 
There are several cases where propositional logic won't help us reach needed conclusions or information; \\
\begin{itemize}
	\item Suppose we know that "All CS students must take CSCI 358". We cannot conclude that "Alice must take CSCI 358 where Alice is a CS student" using our current propositional logic knowledge.
	\item Statements that hold many objects must be enumerated;
	\begin{itemize}
		\item Example:
		\begin{itemize}
			\item If Alice is a CS student, then Alice must take CSCI358.
			\item If Bob is a CS student, then Bob must take CSCI358.
			\item If Chris is a CS student, then Chris must take CSCI358.
			\item ...
		\end{itemize}
		\item Solution: make statements with variables
		\begin{itemize}
			\item If x is a CS student, then x must take CSCI358.
		\end{itemize}
	\end{itemize}
	\item Statements that define the property of a group of objects;
	\begin{itemize}
		\item Example:
		\begin{itemize}
			\item All new cars must be registered.
			\item Some of the new CS students graduate with honor.
		\end{itemize}
		\item Solution: Make statements with quantifiers:
		\begin{itemize}
			\item Universal Quantifier - the property is satisfied by all members of the group.
			\item Existential Quantifier - at least one member of the group satisfies the property.
		\end{itemize}
	\end{itemize}
\end{itemize}

\pagebreak

\subsubsection{Predicate representation}
Predicates are represented like functions in other branches of maths; \\
\sidenote[Once we plug in a value for x, the predicate becomes a proposition]e.g $P(x)$ represents some predicate such as "x is a perfect square". \smallbreak
Note that predicates can involve multiple variables, e.g Q(x,y) is "x is greater than y." \bigbreak

The two main quantifiers are represented with $\forall$ and $\exists$ \\
\begin{itemize}
	\item Universal Quantifier: $\forall$
	\begin{itemize}
		\item Read as “for all,” “for every,” “for each,” or “for any.”
		\item Ex: $\forall x, x>0$ is read as “for any number $x$, $x$ is greater than 0.”
	\end{itemize}
	\item Existential Quantifier: $\exists$
	\begin{itemize}
		\item Read as “there exists one,” “there is,” “for at least one,” or “for some.”
		\item Example: $\exists x, x>0$ is read as “there exists a number x such that x is greater than zero.”
	\end{itemize}
\end{itemize}
\intextnote[\textnormal{When $\forall x P(x)$ or $\exists x P(x)$ is used, the domain must be specified.}]
\bigbreak

\subsubsection*{Truth Values of Predicates}
\smallbreak
\fontsize{9}{10}\selectfont
\begin{tabular}{|p{0.6in} | p{1.3in} | p{1.3in} | p{2.1in} |}
	\hline
	\rowcolor{lightgray} Predicate & True When... & False When... & Examples \\
	\hline \\
	$\forall x P(x)$ & If P(x) is true for \textbf{every} x in the domain & If there is \textbf{any} x in the domain such that P(x) is false & \Longunderstack[l]{$P(x)$ is $x+1>x$, $\forall P(x)$ is true \\ for the domain consisting of all \\ real numbers. \\ ----------------------------------------------- \\ $Q(x)$ is $x <2$. $\forall x Q(x)$ is false \\ for the domain consisting of all real \\ numbers because $Q(3)$ is false. \\ $x=3$ is a counterexample of $\forall x Q(x)$} \\ \\
	\hline \\
	$\exists x P(x)$ & There's is an x \textbf{anywhere} such that P(x) is true. & P(x) is false for \textbf{every} x & \Longunderstack[l]{$P(x)$ is $x > 3$. $\exists x P(x)$ is true \\ for the domain consisting of all real \\ numbers. Because when x=4, P(4) \\ is true.  \\ ----------------------------------------------- \\ $Q(x)$ is $X=x+1$. $exists x Q(x)$ is \\ false for the domain consisting of \\ all real numbers. Because $Q(x)$ is \\ false for every real number x} \\ \\
	\hline
	
\end{tabular}

\normalsize
\bigbreak

The quantifiers $\forall$ and $\exists$ have higher precedence than all logical connectives from propositional logic.

For Example: \\
\hspace*{1cm} $\forall x  P(x) \land Q(x)$ means $(\forall x  P(x)) \land Q(x)$ rather than $\forall x  (P(x) \land Q(x))$

\pagebreak

\subsubsection*{Negating Quantified Expressions}
\bigbreak
\bigbreak
\Large $\neg \forall x P(x) \equiv \exists x \neg P(x)$ 
\normalsize
\begin{itemize}[leftmargin=*, label={}]
	\item Example:
	\begin{itemize}
		\item Every CS Student Must take CSCI385.
		\item Negation: There is a CS student who doesn't have to take CSCI358
	\end{itemize}
\end{itemize}

\bigbreak
\bigbreak

\Large $\neg \exists x P(x) \equiv \forall x \neg P(x)$
\normalsize
\begin{itemize}[leftmargin=*, label={}]
	\item Example:
	\begin{itemize}
		\item There is s student in this class who has taken CSCI 262.
		\item Negation: Every student in this class has \textbf{not} taken CS262
	\end{itemize}
\end{itemize}

\end{document}




