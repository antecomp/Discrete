\documentclass[12pt, letterpaper]{article}

% usepackage aids
	
	% for graphics
	\usepackage{graphicx}
	\graphicspath{{images/}} %configuring the graphicx package

	% to change the margin sizes
	\usepackage[margin=2.7cm]{geometry}

	% assorted maths packages
	\usepackage{xfrac,bigints}

	% handles maths better than latex default
	\usepackage{amsmath}

	% for wrapping text around images
	\usepackage{wrapfig}% http://ctan.org/pkg/wrapfig

	% colour tables
	\usepackage[table]{xcolor}

	% idk
	\usepackage{wasysym}

	% custom margins
	\usepackage{scrextend}

	% for symbols and glyphs
	\usepackage{fontawesome}

	% adds the fancy fun margin notes
	\setlength {\marginparwidth }{2cm}
	\usepackage[colorinlistoftodos]{todonotes}

	% Add linebreaks without indentation because you need a package for that too????
	\usepackage[parfill]{parskip}

	% for highlighting text
	\usepackage{soul}
		% make multicolour not aids
		\newcommand{\hlc}[2][yellow]{{\sethlcolor{#1}\hl{#2}}}

	% ability to mess with bullet list indents
	\usepackage{enumitem}


	% subtable maybe
	\usepackage{booktabs,subcaption,amsfonts,dcolumn}


	% Better underlines
	\usepackage{contour}
	\usepackage[normalem]{ulem}

	\renewcommand{\ULdepth}{1.8pt}
	\contourlength{0.8pt}

	\newcommand{\cul}[1]{%
		\uline{\phantom{#1}}%
		\llap{\contour{white}{#1}}%
	}



	% personal newcommands to make a lot of my formatting easier.
	\newcommand{\exheader}[1][ex]{{\tiny{#1}\normalsize}}

	\newcommand{\sidenote}[1][INSERT TEXT HERE]{{\todo[size=tiny, color=lightgray, nolist]{
			\faInfoCircle\space {#1}
		}
	}}

	\newcommand{\intextnote}[1][Also Note Lorem Ipsum]{{
				\emph{
					\begin{small}
					\begin{quote}
						\faInfoCircle\space
						{#1}
					\end{quote}
					\end{small}
		}
	}}

	\newcommand{\horizline}[0]{\noindent\rule{\textwidth}{1pt}\\}


\title{Discrete Maths Notes}
\begin{document}
\maketitle
\tableofcontents
\pagebreak


\section{Logic}
\bigbreak\bigbreak
\subsection{Propositional Logic}
\bigbreak
\subsubsection{Basics}

\indent
\todo[size=tiny, color=lightgray, nolist]{
	\faInfoCircle\space It's not always easy to determine if they're true/false.
}A \textbf{proposition} is a statement that is either true or false. \\
Prepositions will be represented mathematically with capital letters A, B, C... \\ These prepositions are then are connected into more complex compound prepositions using \emph{connectives}. Connectives are statements like "and, implies, if-then" and are represented mathematically with the symbols below.

\begin{center}
\begin{tabular}{ |p{1.5cm}|p{3.75cm}|p{5.8cm}|p{3.25cm}| }
 \hline
 \multicolumn{4}{|c|}{Connectives} \\
 \hline
 \hline
 \rowcolor{lightgray} Symbol& Name & English Term(s) & Reading \\
 \hline
 	$\land$ 	& 	AND 		& And, But, Also	& A and B  	\\
	\hline
 	$\lor$ 		& 	OR 			& -	& A or B	\\
	\hline
 	$\implies$ 	& 	IMPLICATION	& \begin{itemize}[leftmargin=*, label={}]
		\item If A, Then B
		\item If A, then B
	 	\item A implies B
		\item A, therefore B
		\item A only if B
		\item B follows from A
		\item A is a sufficient condition for B
		\item B is a necessary condition for A 
	\end{itemize} & A implies B \\
	\hline
	$\iff$ 		&	BICONDITIONAL	& If \& only if	 A is necessary and sufficient for B & A if and only if B 	\\
	\hline
	$\neg$ 		&	NEGATION		& Not... & Not A \\
 \hline	
\end{tabular}
\end{center}

\emph{
\begin{small}
\begin{quote}
	\faInfoCircle\space
	A Bicondtional can also be thought of $(A \implies B)\land(B \implies A)$ \\
	Negation may sometimes be represented as $A\prime \textrm{ or } \overline{A}$
\end{quote}
\end{small}
}

\bigbreak

\subsubsection{Terminology}
\begin{addmargin}[2em]{2em}
	$ A \land B $ - \cul{conjuction} of \cul{conjuncts} $A$ and $B$ \\
	$ A \lor B $ - \cul{disjunction} of \cul{disjuncts} $A$ and $B$ \\
	$ A \implies B $ - $A$ is the \cul{hypothesis\tiny{/antecedant}} and $B$ is the \cul{conclusion\tiny{/consequence}}  
\end{addmargin}


\pagebreak
\subsubsection{Examples}

\exheader[1 Compound Proposition]
\begin{addmargin}[1.5em]{1.5em}
	If \cul{all humans are mortal}\tiny{prp $A $}\normalsize \space and \cul{all Greeks are human}\tiny{prp $B$}\normalsize\space
	\\then \cul{all Greeks are moral}\tiny{prp $C$}\normalsize\space can be represented as $A \land B \implies C$
\end{addmargin}

\exheader[2 Negation]
\begin{addmargin}[1.5em]{1.5em}
	\begin{itemize}[label={}, leftmargin=*]
		\item Chocolate is sweet $\rightarrow$ Chocolate is \cul{not} sweet
		\item 
			Peter is tall and thin $\rightarrow$ 
			\todo[size=tiny, color=lightgray, nolist]{
				\faInfoCircle\space Short \cul{and} fat would be incorrect!
			}
			Peter is short \cul{or} fat
		\item The river is shallow or polluted $\rightarrow$ 
		\todo[size=tiny, color=lightgray, nolist]{
			\faInfoCircle\space Not shallow \cul{or} not polluted would be incorrect!
		}
		The river is deep \cul{and} polluted.
	\end{itemize}
\end{addmargin}

\exheader[{3 Implication: \hlc[cyan]{hypothesis} and \hlc[green]{conclusion} }]
\begin{addmargin}[1.5em]{1.5em}
	\begin{itemize}[label={}, leftmargin=*]
		\item If \hlc[cyan]{the rain continues} then \hlc[green]{the river will flood}
		\item A sufficient condition for a \hlc[green]{network failure} is that the \hlc[cyan]{central switch goes down}
		\item \hlc[cyan]{The avocados are ripe} only if \hlc[green]{they are dark and soft}
		\item \hlc[green]{A good diet} is a necessary condition for \hlc[cyan]{a healthy cat}
	\end{itemize}
\end{addmargin}

\bigbreak
\subsubsection{Satifiability, Tautology, Contradiction}
\begin{itemize}[label={}, leftmargin=*]
	\item A proposition is \cul{satisfiable} if it is true for
	\sidenote[You don't need a whole truth table for this, just look for one!]\emph{at least one} 
	combination of boolean values.
	\item \cul{A Boolean Satisfiablity Problem (SAT)} is checking for satifiability in a propositional logic formula.
	\item \cul{A Tautology} is a proposition that is \cul{always true} \\ \hspace*{5mm} \exheader[ex] $ A \lor \neg A$
	\item \cul{A Contradiction} is a proposition that is always false. \\ \hspace*{5mm} \exheader[ex] $ A \land \neg A$
\end{itemize}

\pagebreak

\subsection{Truth Tables}
\bigbreak
\subsubsection{Basics}
Truth Tables are used for determining all the possible outputs of a complex compound propostion.
\\
\horizline
\\
\cul{The Columns} Are for the prepositions,
\sidenote[The intrmt' prepositions are optional steps to make solving easier, use as needed.]intermediate compound prepositions and the whole compound preposition.

\cul{The Rows} Are to contian the different sets of possible truth values for each proposition. You will have $2^p$ rows where $p$ is the number of propositions (then +1 for the header).

\horizline \\
\faWarning \space The connectives in a compound propositional logic problem follow an order of precedence (the PEMDAS of logic) in the following order; \\
$ \neg $ , $ \land $ , $ \lor $ , $ \implies $ , $ \iff $
\\
\subsubsection{Connective Outputs}
\begin{table}[h]
    \begin{subtable}[h]{0.20\textwidth}
        \centering
        \begin{tabular}{l | l}
		\multicolumn{2}{c}{\textbf{Negation}} \\
        \hline \hline
        $A$ & $\neg A$ \\
		\hline
        T & F \\
		F & T
       \end{tabular}
    \end{subtable}
    \hfill
    \begin{subtable}[h]{0.20\textwidth}
        \centering
        \begin{tabular}{l | l | l}
		\multicolumn{3}{c}{\textbf{And}} \\
        \hline \hline
		$A$ & $B$ & $A \land B$ \\
		\hline
		T & T & T \\
		T & F & F \\
		F & T & F \\
		F & F & F 
        \end{tabular}
     \end{subtable}
	 \hfill
	 \begin{subtable}[h]{0.20\textwidth}
        \centering
        \begin{tabular}{l | l | l}
		\multicolumn{3}{c}{\textbf{Or}} \\
        \hline \hline
		$A$ & $B$ & $A \lor B$ \\
		\hline
		T & T & T \\
		T & F & T \\
		F & T & T \\
		F & F & F 
        \end{tabular}
     \end{subtable}

\end{table}



\begin{table}[h]
    \begin{subtable}[h]{0.45\textwidth}
        \centering
        \begin{tabular}{l | l | l}
		\multicolumn{3}{c}{\textbf{Implication}} \\
        \hline \hline
		$A$ & $B$ & $A \implies B$ \\
		\hline
		T & T & T \\
		T & F & F \\
		F & T & T \\
		F & F & T 
        \end{tabular}
		\caption*{\small{An implication is true when the hypothesis is false or the conclusion is true.}}
     \end{subtable}
    \hfill
    \begin{subtable}[h]{0.45\textwidth}
        \centering
        \begin{tabular}{l | l | l}
		\multicolumn{3}{c}{\textbf{Bicondtional}} \\
        \hline \hline
		$A$ & $B$ & $A \iff B$ \\
		\hline
		T & T & T \\
		T & F & F \\
		F & T & F \\
		F & F & T 
        \end{tabular}
		\caption*{\small{A Bicondtional is true when the two prepositions have the same value.}}
     \end{subtable}
\end{table}

\pagebreak
\subsubsection{Examples}

\begin{table}[h]
    \begin{subtable}[h]{0.45\textwidth}
        \centering
        \begin{tabular}{l | l | l | l | l}
		\multicolumn{5}{c}{\textbf{$A \implies B \iff B \implies A$}} \\
        \hline \hline
		$A$ & $B$ & $A \implies B$ & $ B \implies A $ & ------ \\
		\hline
		T & T & T & T & T \\
		T & F & F & T & F \\
		F & T & T & F & F \\
		F & F & T & T & T
        \end{tabular}
     \end{subtable}
    \hfill
    \begin{subtable}[h]{0.45\textwidth}
        \centering
        \begin{tabular}{l | l | l | l | l}
		\multicolumn{5}{c}{\textbf{$A\land \neg B \implies \neg C$}} \\
        \hline \hline
		$A$ & $B$ & C & $A \land \neg B $ & --------  \\
		\hline
			T & T & T & F & T \\
			T & T & F & F & T \\
			T & F & T & T & F \\
			T & F & F & T & T \\
			F & T & T & F & T \\
			F & T & F & F & T \\
			F & F & T & F & T \\
			F & F & F & F & T
        \end{tabular}
     \end{subtable}
\end{table}
\intextnote[Remember, columns like $A \implies B$ are optional in-between steps to help solve each problem.]

\end{document}




