\documentclass[12pt, letterpaper]{article}

% usepackage aids
	
	% for graphics
	\usepackage{graphicx}
	\graphicspath{{images/}} %configuring the graphicx package

	% to change the margin sizes
	\usepackage[margin=2.7cm]{geometry}

	% assorted maths packages
	\usepackage{xfrac,bigints}

	% handles maths better than latex default
	\usepackage{amsmath}

	% for wrapping text around images
	\usepackage{wrapfig}% http://ctan.org/pkg/wrapfig

	% colour tables
	\usepackage[table]{xcolor}

	% idk
	\usepackage{wasysym}

	% custom margins
	\usepackage{scrextend}

	% for symbols and glyphs
	\usepackage{fontawesome}

	% adds the fancy fun margin notes
	\setlength {\marginparwidth }{2cm}
	\usepackage[colorinlistoftodos]{todonotes}

	% Add linebreaks without indentation because you need a package for that too????
	\usepackage[parfill]{parskip}

	% for highlighting text
	\usepackage{soul}
		% make multicolour not aids
		\newcommand{\hlc}[2][yellow]{{\sethlcolor{#1}\hl{#2}}}

	% ability to mess with bullet list indents
	\usepackage{enumitem}


	% subtable maybe
	\usepackage{booktabs,subcaption,amsfonts,dcolumn}


	% Better underlines
	\usepackage{contour}
	\usepackage[normalem]{ulem}

	\renewcommand{\ULdepth}{1.8pt}
	\contourlength{0.8pt}

	\newcommand{\cul}[1]{%
		\uline{\phantom{#1}}%
		\llap{\contour{white}{#1}}%
	}

	% monospace font
	\usepackage{courier}
	\usepackage[T1]{fontenc}



	%%%%%%%%%%%%%%%%%%%%%%%%%%%%%%%%%%%%%%%%%%%%%%%%%%%%%%%%%%%%%%%%%%%%%%%%%%%



	% personal newcommands to make a lot of my formatting easier.
	\newcommand{\exheader}[1][ex]{{\tiny{#1}\normalsize}}

	\newcommand{\sidenote}[1][INSERT TEXT HERE]{{\todo[size=tiny, color=lightgray, nolist]{\faInfoCircle\space {#1}}}}

	\newcommand{\intextnote}[1][Also Note Lorem Ipsum]{{
				\emph{
					\begin{small}
					\begin{quote}
						\faInfoCircle\space
						{#1}
					\end{quote}
					\end{small}
		}
	}}

	\newcommand{\horizline}[0]{\noindent\rule{\textwidth}{1pt}\\}


	% for equiv tables
	\newcommand{\eqtbitem}[3][----]{ {#1} & $ \equiv $ & {#2} & {#3} }


\title{Discrete Maths Notes}
\begin{document}
\maketitle
\tableofcontents
\pagebreak


\section{Logic}
\bigbreak\bigbreak
\subsection{Propositional Logic}
\bigbreak
\subsubsection{Basics}

\indent
\todo[size=tiny, color=lightgray, nolist]{
	\faInfoCircle\space It's not always easy to determine if they're true/false.
}A \textbf{proposition} is a statement that is either true or false. \\
Prepositions will be represented mathematically with capital letters A, B, C... \\ These prepositions are then are connected into more complex compound prepositions using \emph{connectives}. Connectives are statements like "and, implies, if-then" and are represented mathematically with the symbols below.

\begin{center}
\begin{tabular}{ |p{1.5cm}|p{3.75cm}|p{5.8cm}|p{3.25cm}| }
 \hline
 \multicolumn{4}{|c|}{Connectives} \\
 \hline
 \hline
 \rowcolor{lightgray} Symbol& Name & English Term(s) & Reading \\
 \hline
 	$\land$ 	& 	AND 		& And, But, Also	& A and B  	\\
	\hline
 	$\lor$ 		& 	OR 			& -	& A or B	\\
	\hline
 	$\implies$ 	& 	IMPLICATION	& \begin{itemize}[leftmargin=*, label={}]
		\item If A, Then B
		\item If A, then B
	 	\item A implies B
		\item A, therefore B
		\item A only if B
		\item B follows from A
		\item A is a sufficient condition for B
		\item B is a necessary condition for A 
	\end{itemize} & A implies B \\
	\hline
	$\iff$ 		&	BICONDITIONAL	& If \& only if	 A is necessary and sufficient for B & A if and only if B 	\\
	\hline
	$\neg$ 		&	NEGATION		& Not... & Not A \\
 \hline	
\end{tabular}
\end{center}

\emph{
\begin{small}
\begin{quote}
	\faInfoCircle\space
	A Biconditional can also be thought of $(A \implies B)\land(B \implies A)$ \\
	Negation may sometimes be represented as $A\prime \textrm{ or } \overline{A}$
\end{quote}
\end{small}
}

\bigbreak

\subsubsection{Terminology}
\begin{addmargin}[2em]{2em}
	$ A \land B $ - \cul{conjunction} of \cul{conjuncts} $A$ and $B$ \\
	$ A \lor B $ - \cul{disjunction} of \cul{disjuncts} $A$ and $B$ \\
	$ A \implies B $ - $A$ is the \cul{hypothesis\tiny{/antecedent}} and $B$ is the \cul{conclusion\tiny{/consequence}}  
\end{addmargin}


\pagebreak
\subsubsection{Examples}

\exheader[1 Compound Proposition]
\begin{addmargin}[1.5em]{1.5em}
	If \cul{all humans are mortal}\tiny{prp $A $}\normalsize \space and \cul{all Greeks are human}\tiny{prp $B$}\normalsize\space
	\\then \cul{all Greeks are mortal}\tiny{prp $C$}\normalsize\space can be represented as $A \land B \implies C$
\end{addmargin}

\exheader[2 Negation]
\begin{addmargin}[1.5em]{1.5em}
	\begin{itemize}[label={}, leftmargin=*]
		\item Chocolate is sweet $\rightarrow$ Chocolate is \cul{not} sweet
		\item 
			Peter is tall and thin $\rightarrow$ 
			\todo[size=tiny, color=lightgray, nolist]{
				\faInfoCircle\space Short \underline{and} fat would be incorrect!
			}
			Peter is short \cul{or} fat
		\item The river is shallow or polluted $\rightarrow$ 
		\todo[size=tiny, color=lightgray, nolist]{
			\faInfoCircle\space Not shallow \underline{or} not polluted would be incorrect!
		}
		The river is deep \cul{and} polluted.
	\end{itemize}
\end{addmargin}

\exheader[{3 Implication: \hlc[cyan]{hypothesis} and \hlc[green]{conclusion} }]
\begin{addmargin}[1.5em]{1.5em}
	\begin{itemize}[label={}, leftmargin=*]
		\item If \hlc[cyan]{the rain continues} then \hlc[green]{the river will flood}
		\item A sufficient condition for a \hlc[green]{network failure} is that the \hlc[cyan]{central switch goes down}
		\item \hlc[cyan]{The avocados are ripe} only if \hlc[green]{they are dark and soft}
		\item \hlc[green]{A good diet} is a necessary condition for \hlc[cyan]{a healthy cat}
	\end{itemize}
\end{addmargin}

\bigbreak
\subsubsection{Satifiability, Tautology, Contradiction}
\begin{itemize}[label={}, leftmargin=*]
	\item A proposition is \cul{satisfiable} if it is true for
	\sidenote[You don't need a whole truth table for this, just look for one!]\emph{at least one} 
	combination of boolean values.
	\item \cul{A Boolean Satisfiability Problem (SAT)} is checking for satisfiability in a propositional logic formula.
	\item \cul{A Tautology} is a proposition that is \cul{always true} \\ \hspace*{5mm} \exheader[ex] $ A \lor \neg A$
	\item \cul{A Contradiction} is a proposition that is always false. \\ \hspace*{5mm} \exheader[ex] $ A \land \neg A$
\end{itemize}

\pagebreak

\subsection{Truth Tables}
\bigbreak
\subsubsection{Basics}
Truth Tables are used for determining all the possible outputs of a complex compound propostion.
\\
\horizline
\\
\cul{The Columns} Are for the prepositions,
\sidenote[The intrmt' prepositions are optional steps to make solving easier, use as needed.]intermediate compound prepositions and the whole compound preposition.

\cul{The Rows} Are to contain the different sets of possible truth values for each proposition. You will have $2^p$ rows where $p$ is the number of propositions (then +1 for the header).

\horizline \\
\faWarning \space The connectives in a compound propositional logic problem follow an order of precedence (the PEMDAS of logic) in the following order; \\
$ \neg $ , $ \land $ , $ \lor $ , $ \implies $ , $ \iff $
\\
\subsubsection{Connective Outputs}
\begin{table}[h]
    \begin{subtable}[h]{0.20\textwidth}
        \centering
        \begin{tabular}{l | l}
		\multicolumn{2}{c}{\textbf{Negation}} \\
        \hline \hline
        $A$ & $\neg A$ \\
		\hline
        T & F \\
		F & T
       \end{tabular}
    \end{subtable}
    \hfill
    \begin{subtable}[h]{0.20\textwidth}
        \centering
        \begin{tabular}{l | l | l}
		\multicolumn{3}{c}{\textbf{And}} \\
        \hline \hline
		$A$ & $B$ & $A \land B$ \\
		\hline
		T & T & T \\
		T & F & F \\
		F & T & F \\
		F & F & F 
        \end{tabular}
     \end{subtable}
	 \hfill
	 \begin{subtable}[h]{0.20\textwidth}
        \centering
        \begin{tabular}{l | l | l}
		\multicolumn{3}{c}{\textbf{Or}} \\
        \hline \hline
		$A$ & $B$ & $A \lor B$ \\
		\hline
		T & T & T \\
		T & F & T \\
		F & T & T \\
		F & F & F 
        \end{tabular}
     \end{subtable}

\end{table}



\begin{table}[h]
    \begin{subtable}[h]{0.45\textwidth}
        \centering
        \begin{tabular}{l | l | l}
		\multicolumn{3}{c}{\textbf{Implication}} \\
        \hline \hline
		$A$ & $B$ & $A \implies B$ \\
		\hline
		T & T & T \\
		T & F & F \\
		F & T & T \\
		F & F & T 
        \end{tabular}
		\caption*{\small{An implication is true when the hypothesis is false or the conclusion is true.}}
     \end{subtable}
    \hfill
    \begin{subtable}[h]{0.45\textwidth}
        \centering
        \begin{tabular}{l | l | l}
		\multicolumn{3}{c}{\textbf{Biconditional}} \\
        \hline \hline
		$A$ & $B$ & $A \iff B$ \\
		\hline
		T & T & T \\
		T & F & F \\
		F & T & F \\
		F & F & T 
        \end{tabular}
		\caption*{\small{A Biconditional is true when the two prepositions have the same value.}}
     \end{subtable}
\end{table}

Out of all these outputs, the most unintuitive is the 3rd implication output ($F, T \implies T$). The easiest way to understand this output is with the proposition ``If it is raining, then the ground is wet''; now say you step outside and it is not raining, but the ground is wet. The entire statement isn't false or incorrect, but the first part of it still has a false value. \\ The only way to make an implication false is when the hypothesis is true but the conclusion is false.


\pagebreak
\subsubsection{Examples}

\begin{table}[h]
    \begin{subtable}[h]{0.45\textwidth}
        \centering
        \begin{tabular}{l | l | l | l | l}
		\multicolumn{5}{c}{\textbf{$A \implies B \iff B \implies A$}} \\
        \hline \hline
		$A$ & $B$ & $A \implies B$ & $ B \implies A $ & ------ \\
		\hline
		T & T & T & T & T \\
		T & F & F & T & F \\
		F & T & T & F & F \\
		F & F & T & T & T
        \end{tabular}
     \end{subtable}
    \hfill
    \begin{subtable}[h]{0.45\textwidth}
        \centering
        \begin{tabular}{l | l | l | l | l}
		\multicolumn{5}{c}{\textbf{$A\land \neg B \implies \neg C$}} \\
        \hline \hline
		$A$ & $B$ & C & $A \land \neg B $ & --------  \\
		\hline
			T & T & T & F & T \\
			T & T & F & F & T \\
			T & F & T & T & F \\
			T & F & F & T & T \\
			F & T & T & F & T \\
			F & T & F & F & T \\
			F & F & T & F & T \\
			F & F & F & F & T
        \end{tabular}
     \end{subtable}
\end{table}
\intextnote[Remember, columns like $A \implies B$ are optional in-between steps to help solve each problem.]
\bigbreak
\subsubsection{Exercise: Finding Tautologies, Satisfiable \& Contradicting Props'}

Indicate whether each of the following is a tautology, satisfiable but not a tautology or a contradiction;
\begin{enumerate}[leftmargin=*, label={}]
	\item $ A \implies B $
	\item $ A \implies A $
	\item $ A \implies \neg B \lor \neg C $
	\item $ A \lor B \implies B $
	\item $ (A \land B) \implies (A \lor B) $
	\item $ A \lor \neg A \implies B \land \neg B $
\end{enumerate}

\small{(Answers and explanations on the next page...)} \normalsize

\pagebreak

\intextnote[Notice how none of these rely on drawing out a whole truth table! Focus on trying to find a way to get each proposition to output true and a way to get it to output false!]

$ A \implies B $ \\
\emph{Satisfiable but not a tautology} \\
Just knowing the properties of an implication you should know there's way to get true outputs and a false output.

$ A \implies A $ \\
\emph{Tautology} \\
Only would be $ T \implies T $ or $ F \implies F $, both of which result in true.

$ A \implies \neg B \lor \neg C $ \\
\emph{Satisfiable but not a tautology} \\
Instead of making a long unpleasant truth table, it's easiest to start by simply looking for one true and one false possible output. \\
We can make the left side true simply by making A false, since all that remains is an or statement we now have a true output. \\
We can just as easily find a false output for this proposition with $A=T$, $B=T \space (\neg B = F)$ to make the implication false, then we can just make $ \neg C $ false to make the or output false.

$A \lor B \implies B $ \\
\emph{Satisfiable but not a tautology} \\
If we make B true then the biconditional will always be true regardless of A. \\
There is only one way to make an implication false, so if we can set up A and B to result in that false output, it won't be a tautology. If we make A true and B false it will make the implication false!

$ (A \land B) \implies (A \lor B) $ \\
\emph{Tautology} \\
Remember the only way to make an implication false is if the hypothesis is true and the conclusion is false. There is absolutely no way to do this because of the and/or setup! 

$ A \lor \neg A \implies B \land \neg B $ \\
\emph{Contradiction} \\
The left side is always true and the right side is always false. So the result of the implication is always false!
\pagebreak

\subsection{Equivalence}
\bigbreak
\subsubsection{Introduction to Equivalence}

\horizline
\faBook \space Two (compound) propositions $P$ and $Q$ are \textbf{logically equivalent} when their truth values always match (Meaning they'll have the same truth table!). \\
Equivalence is denoted by $P \equiv Q$.\\
\horizline
\\
Equivalence relates heavily to the concept of Tautologies;
\begin{itemize}[label={}, leftmargin=*]
	\item $P$ and $Q$ are equivalent when $P \iff Q$ is a tautology.
	\item A proposition $P$ is a tautology iff (if and only if) it is equivalent to $T$ (true), i.e $P \equiv T$
\end{itemize}

\bigbreak

\subsubsection*{Examples}
\exheader[1]\\
Given the implication $A \implies B$, are the following equivalent?
\begin{itemize}[leftmargin=*, label={}]
	\item The contrapositive: $\neg B \implies \neg A$
	\item The converse: $B \implies A$
\end{itemize}
\bigbreak
\begin{center}
\begin{tabular}{l | l || l | l | l}
	\hline 
	$A$ & $B$ & $A \implies B$ & $ \neg B \implies \neg A $ & $B \implies A$ \\
	\hline
	T & T & T & T & T \\
	T & F & F & F & T \\
	F & T & T & T & F \\
	F & F & T & T & T
	\end{tabular} 
\end{center} \bigbreak
Looking at the table we can see that $A \implies B$ and $\neg B \implies \neg A$ are equivalent.
\bigbreak


Now, what about $\neg A \lor B $ ? \\
\begin{center}
\begin{tabular}{l | l || l | l}
	\hline
	$A$ & $B$ & $A \implies B$ & $\neg A \lor B $ \\
	\hline
	T & T & T & T \\
	T & F & F & F \\
	F & T & T & T \\
	F & F & T & T \\
\end{tabular}
\end{center} \bigbreak
\indent \sidenote[This is actually one of the equivalence laws you'll see in the next \\ section!]Yep! $\neg A \lor B \equiv A \implies B$.

\pagebreak


\exheader[2: Code Logic Optimization]

Understanding equivalent boolean expressions is very important in computer science (for code) and chip design (for logic gates). Consider the code below;
\begin{verbatim}
	if(x > 0 || (x <= 0 && y > 100))
\end{verbatim}
Lets see if we can change this expression to something equivalent but simplified. \\
Let $A$ be \texttt{x > 0} and let $B$ be \texttt{y > 100} \\
Now we can compare the truth values of $ A \lor (\neg A \land B) $ and $ A \lor B $. \\
\begin{center}
	\begin{tabular}{l | l || l | l}
		\hline
		$A$ & $B$ & $A \lor (\neg A \land B)$ & $ A \lor B $ \\
		\hline
		T & T & T & T \\
		T & F & T & T \\
		F & T & T & T \\
		F & F & F & F \\
	\end{tabular}
\end{center} \bigbreak
They're equivalent! We can reduce the if statement's expression to simply;
\begin{verbatim}
	if(x > 0 || y > 100)
\end{verbatim}

\pagebreak
\subsubsection{Equivalence Laws}
For more complex propositions it is impractical to create a set of massive truth tables to check for equivalence. So instead we utilize equivalence laws to directly prove equivalence.
\bigbreak
\subsubsection*{Nine Equivalence Laws;}
\small{\emph{Many of these are pretty self-explanatory}}\normalsize
\begin{itemize}[label={}, leftmargin=*]
	\item Double Negation Law: $\neg(\neg A) \equiv A$ 
	\item Identity Laws: $ A \land T \equiv A $ \hspace*{1.5em} $ A \lor F \equiv A$
	\item Domination Laws: $ A \lor T \equiv T $ \hspace*{1.5em} $ A \land F \equiv F $
	\item Commutative Laws: $ A \land B \equiv B \land A$  \hspace*{1.5em} $ A \lor B \equiv B \lor A $
	\item Associative Laws: $ (A \land B) \land C \equiv A \land (B \land C) $ \hspace*{1.5em} $(A \lor B) \lor C \equiv A \lor (B \lor C) $
	\item Idempotent Laws: $ A \land A \equiv A $ \hspace*{1.5em} $ A \lor A \equiv A$
	\item\sidenote[Very similar to the algebriac distributive law]Distributive Laws: $A \lor (B \land C) \equiv (A \lor B) \land (A \lor C) $ \hspace*{1.5em} $A \land (B \lor C) \equiv (A \land B) \lor (A \land C) $
	\item DeMorgan's Laws: $\neg (A \land B) \equiv \neg A \lor \neg B $ \hspace*{1.5em} $\neg (A \lor B) \equiv \neg A \land \neg B$
	\item Implication Laws: $ A \implies B \equiv \neg B \implies \neg A \equiv \neg A \lor B$
\end{itemize}
\todo[inline]{TODO: Maybe refomat this as a table so its a bit easier to quickly reference?}
% in solving these, our goal is to reduce the # of letters in the props. - do the side with more stuff going on :)

\pagebreak

\subsubsection*{Examples}
\exheader[1]
Prove $A \lor (\neg A \land B) \equiv A \lor B$ \\ \break
\begin{tabular}{r  l  l  l}
	\eqtbitem[$ A \lor (\neg A \land B) $]{$(A \lor \neg A)\land(A \lor B)$}{(Distributive)} \\
	\eqtbitem[]{$T \land (A \lor B)$}{} \\
	\eqtbitem[]{$A \lor B$}{(Identity)}
\end{tabular}
\intextnote[In solving these, the goal should be to reduce the \# of letters in the propositions. Focus on the side of an equivalence with more going on and try to reduce it down since the more complex proposition will have more oppurtunities to utilize the different equivalence laws. ]
\bigbreak\bigbreak


\exheader[2]
Simplify $ A \land \neg(A \land B) $ \\
\begin{tabular}{r  l  l  l}
	\eqtbitem[$ A \land \neg(A \land B) $]{$A \land (\neg A \lor \neg B)$}{(DeMorgan's)} \\
	\eqtbitem[]{$(A \land \neg A) \lor (A \land \neg B)$}{(Distributive)} \\
	\eqtbitem[]{$F \lor (A \land \neg B)$}{} \\
	\eqtbitem[]{$A \land \neg B$}{(Identity)}
\end{tabular}
\intextnote[You don't have to name the laws you're using in the homework, the simple $\equiv$ down the middle format for each step is fine.]
\bigbreak\bigbreak

\exheader[3] Show that $(A \land B) \implies (A \lor B)$ is a tautology. \\
\begin{tabular}{r  l  l  l}
	\eqtbitem[$(A \land B) \implies (A \lor B)$]{$\neg (A \land B) \lor (A \lor B)$}{(Implication)} \\
	\eqtbitem[]{$(\neg A \lor \neg B)\lor (A\lor B)$}{(DeMorgan's)} \\
	\eqtbitem[]{$\neg A \lor \neg B \lor A \lor B$}{(Associative)} \\
	\eqtbitem[]{$\neg A \lor A \lor \neg B \lor B$}{(Commutative)} \\
	\eqtbitem[]{$(\neg A \lor A) \lor (\neg B \lor B)$}{(Associative)} \\
	\eqtbitem[]{$T \lor T$}{} \\
	\eqtbitem[]{$T$}{(Idempotent)}
\end{tabular}

% Todo: work on excercises and add them! (try them yourself before simply copying down!!)

\end{document}




